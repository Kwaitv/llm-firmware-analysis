\documentclass[conference,compsoc]{IEEEtran}

\ifCLASSOPTIONcompsoc
  \usepackage[nocompress]{cite}
\else
  \usepackage{cite}
\fi

\ifCLASSINFOpdf
  % \usepackage[pdftex]{graphicx}
  % declare the path(s) where your graphic files are
  % \graphicspath{{../pdf/}{../jpeg/}}
  % and their extensions so you won't have to specify these with
  % every instance of \includegraphics
  % \DeclareGraphicsExtensions{.pdf,.jpeg,.png}
\else
  % or other class option (dvipsone, dvipdf, if not using dvips). graphicx
  % will default to the driver specified in the system graphics.cfg if no
  % driver is specified.
  % \usepackage[dvips]{graphicx}
  % declare the path(s) where your graphic files are
  % \graphicspath{{../eps/}}
  % and their extensions so you won't have to specify these with
  % every instance of \includegraphics
  % \DeclareGraphicsExtensions{.eps}
\fi
\usepackage{graphicx}

\hyphenation{op-tical net-works semi-conduc-tor}


\begin{document}
\title{Large Language Models for Binary Firmware Analysis}


\author{\IEEEauthorblockN{Khai Vu}
\and
\IEEEauthorblockN{Pranjal Trivedi}
}

\maketitle

\begin{abstract}
Raw binary analysis is a critical task in cyber-security. However, traditional
  techniques still require cyber-security experts to manually reason through the
  purpose of even if presented in a decompiled state. This is particularly true
  in the case of micro-controller firmware and the assembly associated with
  various peripherals. Large Language Models (LLMs) have demonstrated their
  ability to parse and understand human-readable code. This project seeks to
  extend this notion programmatic understanding beyond human-readable codes
  towards that of compiled binaries, particularly in the embedded space with
  hardware peripherals not explicitly delineated in the ISA of a microprocessor.

\end{abstract}
